%
% >>> Nomenclatures
%
\chapter{Glossary}
%% Symbols
\nomenclatureitem[\textbf{Unit (SI)}]{\textbf{Symbol}}{\textbf{Description}}
\nomenclatureitem[$\Unit{mm}$]{$a$}{Cylindrical Shell Radius}
\nomenclatureitem[$\Unit{mm}$]{$d$}{Tether Diameter}
\nomenclatureitem[$\Unit{mm}$]{$D$}{Outer Drum Diameter}
\nomenclatureitem[$\Unit{GPa}$]{$E$}{Young's Modulus}
\nomenclatureitem[$\Unit{mm}$]{$h$}{Cylindrical Thickness}
\nomenclatureitem[$\Unit{mm^4}$]{$I$}{Moment of Inertia}
\nomenclatureitem[$\Unit{mm}$]{$L$}{Drum Length}
\nomenclatureitem[$\Unit{MPa}$]{$L'$}{Critical Buckling Length}
\nomenclatureitem[$\Unit{-}$]{$n$}{Safety Factor}
\nomenclatureitem[$\Unit{N}$]{$N$}{Normal Force}
\nomenclatureitem[$\Unit{N \cdot mm}$]{$M$}{Moment}
\nomenclatureitem[$\Unit{MPa}$]{$p$}{Pressure}
\nomenclatureitem[$\Unit{MPa}$]{$p'$}{Critical Buckling Pressure}
\nomenclatureitem[$\Unit{mm}$]{$R$}{Outer Drum Radius}
\nomenclatureitem[$\Unit{mm}$]{$t$}{Thickness}
\nomenclatureitem[$\Unit{N}$]{$T$}{Tether Tension}
\nomenclatureitem[$\Unit{mm}$]{$u$}{$x$ Displacement}
\nomenclatureitem[$\Unit{N}$]{$V$}{Shear Force}
\nomenclatureitem[$\Unit{mm}$]{$w$}{$z$ Displacement}
%Greek
\nomenclatureitem[$\Unit{-}$]{$\beta$}{Thin Shell Parameter}
\nomenclatureitem[$\Unit{-}$]{$\varepsilon$}{Strain}
\nomenclatureitem[$\Unit{Radians}$]{$\theta$}{Contact Angle}
\nomenclatureitem[$\Unit{-}$]{$\lambda$}{Load Factor}
\nomenclatureitem[$\Unit{-}$]{$\mu$}{Coefficient of Static Friction}
\nomenclatureitem[$\Unit{-}$]{$\nu$}{Poisson's Ratio}
\nomenclatureitem[$\Unit{MPa}$]{$\sigma$}{Stress}
\nomenclatureitem[$\Unit{MPa}$]{$\sigma_a$}{Allowable Stress}
\nomenclatureitem[$\Unit{MPa}$]{$\sigma_h$}{Hoop Stress}
\nomenclatureitem[$\Unit{MPa}$]{$\sigma_y$}{Yield Strength}
\nomenclatureitem[$\Unit{Radians}$]{$\varphi$}{Differential Angle}
\nomenclatureitem[$\Unit{-}$]{$\psi$}{Buckling Mode}
\nomenclatureitem[]{}{}

%% Acronym
\nomenclatureitem{\textbf{Acronym}}{\textbf{Description}}
\nomenclatureitem{ASME}{American Society of Mechanical Engineers}
\nomenclatureitem{BC}{Boundary Conditions}
\nomenclatureitem{BPVC}{Boiler and Pressure Vessel Code}
\nomenclatureitem{DNV}{Det Norske Veritas}
\nomenclatureitem{FEA}{Finite Element Analysis}
\nomenclatureitem{ID}{Inner Diameter}
\nomenclatureitem{OD}{Outer Diameter}
\nomenclatureitem{ODE}{Ordinary Differential Equation}
\nomenclatureitem{SF}{Safety Factor}
\nomenclatureitem{SI}{International System of Units}
\nomenclatureitem{TMS}{Tether Management System}
\nomenclatureitem{TWPV}{Thin Walled Pressure Vessel}
\nomenclatureitem{USC}{United States Customary}


%% >>> Summary
%%
\chapter{Summary}

The objective of this report is to perform a structural analysis on Altaeros' grounded TMS winch, focusing on the sizing of drum and flange thicknesses.\\

An external pressure $p=1376\Unit{psi}= 9.486\Unit{MPa}$ is calculated from the Capstan equation as a result of the tethers tension. Initial drum thicknesses are determined by study of ASME's BPVC, EN, DNV and TWPV standards yielding a range of $t\in [0.825, 1.680]$ in / $\in [21.0, 43.2]$ mm.\\

The thin shell theory is presented to setup analytical results of fixed end cylinder loaded with uniform pressure $p$ yielding a drum thickness of 1.686 in / 42.8 mm. Buckling analysis is also completed, resulting in 0.418 in / 10.6 mm for a critical buckling pressure of $p'=p$, confirming that buckling is not the primary mode of failure.\\

%%%%%
A series of ANSYS FEA simulations were completed to understand the equivalent stresses for various loading scenarios (focus on USC units). Run 1 explored uniform pressure $p$ with both fixed and simply supported ends yielding required thicknesses of 1.200 \& 0.875 in respectively. Run 2 focused on modeling the exponential decay as per Capstan equation with fixed and simply supported ends yielding 0.650 \& 0.550, respectively. Run 3 focused on a linear eigenvalue buckling analysis to confirm analytical findings. A critical thickness of 0.375 in was yielded. Run 4 served as a preliminary benchmark for flange design. An optimal flange thickness is 0.250 in was observed.\\

In summary, the drum barrel shall be manufactured with a Schedule 30 ($t=$ 0.625 in), 28 in OD pipe. Flanges should be cut from a $3\Unit{feet}^2$ by $0.250\Unit{in}$ thick plate.\\

Future work should focus on flange design, non linear buckling, fatigue analysis and proof testing.
%
% >>> Nomenclatures
%
\chapter{Glossary}
\section*{Symbols}
\nomenclatureitem[\textbf{Unit (SI)}]{\textbf{Symbol}}{\textbf{Description}}
%\nomenclatureitem[$\Unit{UN}$]{$SYMB$}{DESC}
% Non greek A B C D E F G H I J K L M N O P Q R S T U V W X Y Z
\nomenclatureitem[$\Unit{mm}$]{$a$}{Cylindrical Shell Radius}
\nomenclatureitem[$\Unit{mm}$]{$d$}{Tether Diameter}
\nomenclatureitem[$\Unit{mm}$]{$D$}{Outer Drum Diameter}
\nomenclatureitem[$\Unit{GPa}$]{$E$}{Young's Modulus}
\nomenclatureitem[$\Unit{mm}$]{$h$}{Cylindrical Thickness}
\nomenclatureitem[$\Unit{mm^4}$]{$I$}{Moment of Inertia}
\nomenclatureitem[$\Unit{mm}$]{$L$}{Drum Length}
\nomenclatureitem[$\Unit{MPa}$]{$L'$}{Critical Buckling Length}
\nomenclatureitem[$\Unit{-}$]{$n$}{Safety Factor}
\nomenclatureitem[$\Unit{N}$]{$N$}{Normal Force}
\nomenclatureitem[$\Unit{N \cdot mm}$]{$M$}{Moment}
\nomenclatureitem[$\Unit{MPa}$]{$p$}{Pressure}
\nomenclatureitem[$\Unit{MPa}$]{$p'$}{Critical Buckling Pressure}
\nomenclatureitem[$\Unit{mm}$]{$R$}{Outer Drum Radius}
\nomenclatureitem[$\Unit{mm}$]{$t$}{Thickness}
\nomenclatureitem[$\Unit{N}$]{$T$}{Tether Tension}
\nomenclatureitem[$\Unit{mm}$]{$u$}{$x$ Displacement}
\nomenclatureitem[$\Unit{N}$]{$V$}{Shear Force}
\nomenclatureitem[$\Unit{mm}$]{$w$}{$z$ Displacement}



%Greek
\nomenclatureitem[$\Unit{-}$]{$\beta$}{Thin Shell Parameter}
\nomenclatureitem[$\Unit{-}$]{$\varepsilon$}{Strain}
\nomenclatureitem[$\Unit{Radians}$]{$\theta$}{Contact Angle}
\nomenclatureitem[$\Unit{-}$]{$\lambda$}{Load Factor}
\nomenclatureitem[$\Unit{-}$]{$\mu$}{Coefficient of Static Friction}
\nomenclatureitem[$\Unit{-}$]{$\nu$}{Poisson's Ratio}
\nomenclatureitem[$\Unit{MPa}$]{$\sigma$}{Stress}
\nomenclatureitem[$\Unit{MPa}$]{$\sigma_a$}{Allowable Stress}
\nomenclatureitem[$\Unit{MPa}$]{$\sigma_h$}{Hoop Stress}
\nomenclatureitem[$\Unit{MPa}$]{$\sigma_y$}{Yield Strength}
\nomenclatureitem[$\Unit{Radians}$]{$\varphi$}{Differential Angle}
\nomenclatureitem[$\Unit{-}$]{$\psi$}{Buckling Mode}


\section*{Abbreviations}
\nomenclatureitem{\textbf{Acronym}}{\textbf{Description}}
\nomenclatureitem{ASME}{American Society of Mechanical Engineers}
\nomenclatureitem{BC}{Boundary Conditions}
\nomenclatureitem{BPVC}{Boiler and Pressure Vessel Code}
\nomenclatureitem{CAD}{Computer Aided Design}
\nomenclatureitem{DNV}{Det Norske Veritas}
\nomenclatureitem{FEA}{Finite Element Analysis}
\nomenclatureitem{ID}{Inner Diameter}
\nomenclatureitem{OD}{Outer Diameter}
\nomenclatureitem{ODE}{Ordinary Differential Equation}
\nomenclatureitem{SF}{Safety Factor}
\nomenclatureitem{SI}{International System of Units}
\nomenclatureitem{TWPV}{Thin Walled Pressure Vessel}
\nomenclatureitem{USC}{United States Customary}


%% >>> Summary
%%
\chapter{Summary}

As a result of Altaeros' decision to manufacture the grounded tether management's winch system in house, a complete structural analysis is required to be completed. First, the derivation of the Capstan equation was explored to understand how tether tension of $T=11,525 \Unit{lbs} = 51,264\Unit{N}$ translates to external pressure. From this, a pressure of $p=1376\Unit{psi}= 9.486\Unit{MPa}$ was calculated. From this, existing codes and standards were studied to find an initial benchmark. Upon studying ASME's BPVC, EN, DNV and TWPV a range of thicknesses $t\in [0.825, 1.680]$ (inch) were calculated. \\

The governing ODE for thin shell theory was presented and from this, an in depth analysis was completed. Assuming a fixed end cylinder loaded with uniform pressure $p$ returned a thickness of 1.686 in / 42.8 mm matching ASME BPVC VIII-1 calculations. Further buckling analysis was completed to explore other failure modes. Setting the critical buckling pressure to $p'=p$ yielded a critical thickness  of 0.418 in / 10.6 mm, confirming that buckling is not the primary mode of failure.\\

A series of ANSYS FEA runs were completed to understand the equivalent stresses. Run 1 explored uniform pressure $p$ with both fixed and simply supported ends yielding required thicknesses of 1.200 and 0.875 in respectively. Run 2 focused on modeling the exponential decay as per Capstan equation. with simply supported ends yielding a $t=0.550\Unit{in}$. Run 3 focused on a linear eigenvalue buckling analysis to confirm analytical findings. A critical thickness of approximately 0.400 in, converging with preliminary calculations. Run 4 served as a preliminary benchmark for flange design. It was determined that the optimal flange thickness is 0.250 in.\\

In short, based on a tabular summary of all report findings, it was concluded that the winch drum should be manufactured from a Schedule 30 ($t=$ 0.625 in), 28 in OD pipe. Numerical error, material imperfections and non linear buckling were discussed.



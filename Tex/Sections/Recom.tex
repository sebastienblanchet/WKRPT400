Improving testing methods and data extrapolation could be highly beneficial as this could allow more
accurate torque values. Due to the lack of data acquisition flexibility, there is possibly a large error in the
analyzed data. For instance, the extrapolation methods with AutoCAD were approximated as best as
humanly possible. Not to mention the error that is already included in the raw measurements themselves.
Given that the extrapolated data is valid, there are other variables to be factored into the equation.
Mechanical properties such as steel rule curvature, thickness are likely to be similar for the measured
samples but could likely vary for other trials in the future. Hence, the observed torque used to calculate SFs
do not represent the entire operation. Theses inconsistencies are normally a result of coiling the steel rule,
which varies for all coils. Also, further research could be completed with regards to another possible
gearbox solution which does not require any modification to the system. Possible ways to improve the
retrofitting process could be to let a more experienced supplier (Brock Solutions) also deal with the selection
of the gearbox, given ODI’s data.
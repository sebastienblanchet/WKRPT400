As a result of Altaeros' decision to manufacture the grounded TMS winch system, a complete structural analysis is required. First, the derivation of the Capstan equation was explored to understand how tether tension translates to external pressure yielding $p=1376\Unit{psi}= 9.486\Unit{MPa}$ was calculated. Next, existing codes and standards were studied as an initial benchmark for the drum barrel thickness. Study of ASME's BPVC, EN, DNV and TWPV, yielded $\in [0.825, 1.680]$ in / $\in [21.0, 43.2]$ mm.\\

The governing ODE for thin shell theory was presented and from this, an in-depth analysis was completed to determine the drum barrel thickness. Assuming a fixed end cylinder loaded with uniform pressure $p \therefore t = 1.686\Unit{in}=42.8\Unit{mm}$, matching ASME BPVC VIII-1 calculations. Further buckling analysis was completed to explore other failure modes. Setting the critical buckling pressure $p'=p$ yielded a thickness of 0.418 in / 10.6 mm, confirming that buckling is not the primary mode of failure.\\

A series of ANSYS FEA simulations were completed to understand the equivalent stresses for various loading scenarios (focus on USC units). Run 1 explored uniform pressure $p$ with both fixed and simply supported ends yielding required thicknesses of 1.200 \& 0.875 in respectively. Run 2 focused on modeling the exponential decay as per Capstan equation with fixed and simply supported ends yielding 0.650 \& 0.550, respectively. Run 3 focused on a linear eigenvalue buckling analysis to confirm analytical findings. A critical thickness of 0.375 in was yielded. Run 4 served as a preliminary benchmark for flange design. An optimal flange thickness is 0.250 in was observed.\\

In short, based on a tabular summary of all report findings, it was concluded that the winch drum should be manufactured from a Schedule 30 ($t=$ 0.625 in), 28 in OD pipe. Further work was suggested for flange design however, preliminary findings suggest the use of a $3\Unit{feet}\times 3\Unit{feet}$ square $0.250\Unit{in}$ thick plate. Numerical error, material imperfections and nonlinear buckling were also briefly discussed.
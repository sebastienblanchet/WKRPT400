%In conclusion, ODI must improve the DBM’s existing MBA. The 100: 1 reduction Mijno MNT-115-100
%gearbox (GB) and Allen Bradley (AB) N-4220-2-H00AA MBS are highly outdated.\\
%
%Multiple  high by 2. thick steel rule samples were bent 45 and from this current, speed and
%time data were collected. These data were converted to torques. Upon calculation of safety factors (SF)
%with maximum calculated nominal 280and acceleration 345.2  torques both the MBS and GB
%failed SF the 1.25 SF requirement.\\
%
%An ODI supplier for PLC control systems, Brock Solutions was consulted and the new MBS was selected
%to be an AB MPM-A1153F-MJ72AA, rated for 6. nominal, 19. acceleration torque.
%As no GB was suggested, the selection process was left to ODI. Constraints were mechanical fit into the
%DBM and gear ratios between 50 and 100. Existing, GB, GBC , GBM were studied for proper mechanical
%fit. A possible solution was to keep the same  shaft size. Of twelve GBs, none passed the same SF
%requirements.\\
%
%Another solution was to increase the shaft size to . The same procedure was followed and the
%WITTENSTIEN SP-140S-MF2-70-1E1 was selected upon passing aforementioned SF requirements. This
%70:1 reduction GB is rated for 360. and 660. nominal and acceleration torques, respectively.
%To implement this solution, it is required to replace the GBM, bore and broach the GBC and replace the
%socket head cap screws. The proposed design was validated.

Results from Section 2: Codes\\

Results from analytical calculations.\\

Results from FEA.\\

Discussion points.\\
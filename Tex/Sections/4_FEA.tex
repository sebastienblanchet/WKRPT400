In this section, the results from previous sections will be compared with those in an ANSYS Mechanical finite element analysis (FEA) software \cite{ANSYS}. Please note that all results presented in the foregoing section will utilize USC units. Important results will be converted to their respective SI units.

\section{Model}

Using the $D, L$ parameters from Table~\ref{table:prelim_params}, the drum geometry was modeled as a thin surface to save computational cost and time. The geometry and coordinate system adopted in this preceding analysis are shown below in Figure~\ref{fig:4_geom}.

\begin{figure}[H]
	\centering
	\includegraphics[scale=0.5]{4_geom}
	\caption{Geometry and coordinate system overview.}
	\label{fig:4_geom}
\end{figure}

In this model, the drum's thin surface has a parameterized inward thickness. Parametric studies will be performed in ANSYS with this variable to see how maximum stress and deformation vary. The thin surface also allows for a simply initial coarse mesh generated in ANSYS (see Figure~\ref{fig:4_mesh}).

\begin{figure}[H]
	\centering
	\includegraphics[scale=0.5]{4_mesh}
	\caption{ANSYS generated coarse mesh.}
	\label{fig:4_mesh}
\end{figure}

To assure numerically accurate results, a mesh refinement study will be performed on both maximum total deformation and equivalent (Von-Misses) stress for all foregoing simulations.



\section{Run 1: Uniform Pressure}
\label{section:4_R1}
The initial FEA run will have a simplistic case. 

\subsection{Boundary Conditions}
\label{subsection:R1BC}
Both of the drum's ends are fixed (i.e. BCs as per \ref{eq:2_fixedBC}).\\

Figure~\ref{fig:4_R1_p} shows uniform external pressure of $1376 \Unit{psi}$ or $9.484 \Unit{MPa}$ as per \ref{eq:2_preq}.
\begin{figure}[H]
	\centering
	\includegraphics[scale=0.5]{4_R1_p}
	\caption{Uniform external pressure on model.}
	\label{fig:4_R1_p}
\end{figure}

\subsection{Results}

After a mesh refinement, total deformation results are shown below in Figure~\ref{fig:4_R1_def}. The maximum deformation seen on the top surface of the drum in the $y$ direction with respect to $z$ are also shown in Figure~\ref{fig:4_R1_topdef}. Note results shown for $t=0.50\Unit{in}$

\begin{figure}[H]
	\centering
	\includegraphics[scale=0.5]{4_R1_def}
	\caption{Total deformation results.}
	\label{fig:4_R1_def}
\end{figure}
\begin{figure}[H]
	\centering
	\includegraphics[scale=0.6]{4_R1_topdef}
	\caption{Total deformation in $y$ as a function of $z$.}
	\label{fig:4_R1_topdef}
\end{figure}

Comments, lots of deformation.\\

Again, post mesh refinement yields the maximum stress results as per Figure~\ref{fig:4_R1_stress}.

\begin{figure}[H]
	\centering
	\includegraphics[scale=0.5]{4_R1_stress}
	\caption{Total stress results.}
	\label{fig:4_R1_stress}
\end{figure}

Comments, bad stress.


\subsection{Parametric Study}

A parametric study was ran with $t \in [0.50, 1.75]$. This sweep results are shown below in Figure~\ref{fig:4_R1_sweep} \cite{EXCEL}. The allowable stress limit of Equation~\ref{eq:2_sigall} of $23,333 \Unit{psi}$ is also shown for reference.

\begin{figure}[H]
	\centering
	\includegraphics[]{4_R1_sweep}
	\caption{Results of parametric study for FEA Run 1.}
	\label{fig:4_R1_sweep}
\end{figure}

From the above plot, it appears as though a thickness of $t \approx 1.25 \ in$ would appear to be sufficient.

\section{Run 2: Capstan Pressure}
\label{section:4_R2}
The second run continues on the results determined from Run 1. The results presented in this foregoing section will focus on a barrel thickness of $0.50\ in$.

\subsection{Boundary Conditions}

Both of the drum's ends are simply supported (i.e. BCs as per \ref{eq:2_endBC} ). In other words, simple supports carry no end moments.\\

Figure~\ref{fig:4_R2_tens} shows the location of applied tether tension of $11,525\ lbs.$ or $51,264\ N$.
\begin{figure}[H]
	\centering
	\includegraphics[scale=0.5]{4_R2_tens}
	\caption{Remote force location in ANSYS model.}
	\label{fig:4_R2_tens}
\end{figure}

Figure~\ref{fig:4_R2_pvar} shows the variable Capstan pressure applied as per results from \ref{subsection:alt} (see Figure~\ref{fig:4_R2_pvarplot} for $p(z)$ ). Note again that the boundary conditions are for $\mu=0.05$ , which is essentially the worst case scenario. Note results shown for $t=0.50\ in$

\begin{figure}[H]
	\centering
	\includegraphics[scale=0.5]{4_R2_pvar}
	\caption{Variable Capstan pressure.}
	\label{fig:4_R2_pvar}
\end{figure}
\begin{figure}[H]
	\centering
	\includegraphics[scale=0.6]{4_R2_pvarplot}
	\caption{Variable Capstan pressure as a function of longitudinal coordinate $z$.}
	\label{fig:4_R2_pvarplot}
\end{figure}


\subsection{Results}

After a mesh refinement, total deformation results are shown below in Figure~\ref{fig:4_R2_def_mu05}. The maximum deformation seen on the top surface of the drum in the $y$ direction with respect to $z$ are also shown in Figure~\ref{fig:4_R2_topdefplotmu05}.

\begin{figure}[H]
	\centering
	\includegraphics[scale=0.5]{4_R2_def_mu05}
	\caption{Total deformation results.}
	\label{fig:4_R2_def_mu05}
\end{figure}
\begin{figure}[H]
	\centering
	\includegraphics[scale=0.6]{4_R2_topdefplotmu05}
	\caption{Total deformation in $y$ as a function of $z$.}
	\label{fig:4_R2_topdefplotmu05}
\end{figure}

Comments, lots of deformation.\\

Again, post mesh refinement yields the maximum stress results as per Figure~\ref{fig:4_R2_stress_mu05}.

\begin{figure}[H]
	\centering
	\includegraphics[scale=0.5]{4_R2_stress_mu05}
	\caption{Total stress results.}
	\label{fig:4_R2_stress_mu05}
\end{figure}

Comments, bad stress.


\subsection{Parametric Study}

A parametric study was ran with $0.15 \leq t \leq 1.05$ and $0.05 \leq \mu \leq 0.50$. This sweep results are shown below in Figure~\ref{fig:4_R2_sweep} \cite{EXCEL}.  The allowable stress limit of Equation~\ref{eq:2_sigall} of $23,333 \Unit{psi}$ is also shown for reference.

\begin{figure}[H]
	\centering
	\includegraphics[]{4_R2_sweep}
	\caption{Results of parametric study for FEA Run 2.}
	\label{fig:4_R2_sweep}
\end{figure}

From the above plot, it appears as though a thickness of $t \geq 0.50\ in$ would appear to be sufficient.


\section{Run 3: Eigenvalue Buckling}
\label{section:4_R3}
Very unconservative \\

Buckling mode shapes do not represent actual displacements but help you to visualize how a part or an assembly deforms when buckling. \cite{ANSYS}. \\

Load factors $\lambda$ highly dependent on pre-stressed state, either linear on non linear.\\

In this case linear buckling is utilized for its simplicity. Must have coupled static structural analysis (see Figure~\ref{fig:4_R3_wb}.

\begin{figure}[H]
	\centering
	\includegraphics[scale=0.5]{4_R3_wb}
	\caption{Workbench project schematic.}
	\label{fig:4_R3_wb}
\end{figure}

ANSYS takes in the pre-stressed state and returns load factors $\lambda$ which are defined as follows in Equation~\ref{eq:4_loadfactor}.
\begin{equation}
	\label{eq:4_loadfactor}
	p' = \lambda \ p_0
\end{equation}

For this reason, a unit load pressure of $p_0 = 1\ psi$ will be applied. The boundary conditions will follow those of Section~\ref{subsection:R1BC}.

\subsection{Boundary Conditions}

Boundary conditions of coupled static structural analysis. Ends simply supported (Equation~\ref{eq:2_endBC}) and a unit pressure load applied to external surface (see Figure~\ref{fig:4_R3_BC}). Furthermore, for comparison of critical buckling calculations in Section~\ref{section:3_buckle}, a thickness of $t=0.418\Unit{in}$ was modeled. 

\begin{figure}[H]
	\centering
	\includegraphics[scale=0.5]{4_R3_BC}
	\caption{Boundary conditions for Eigenvalue Buckling analysis.}
	\label{fig:4_R3_BC}
\end{figure}

\subsection{Results}

Unfortunately, ANSYS is unable to perform a mesh refinement study for a coupled static structural and eigenvalue buckling analysis hence, a much finer initial mesh than that shown in Figure~\ref{fig:4_mesh} was used to assure reasonably accurate results on the first iteration.\\

Results are shown below for the first eigenvalue or mode $\psi = 1$ below in Figure~\ref{fig:4_R3_mode1}.
\begin{figure}[H]
	\centering
	\includegraphics[scale=0.5]{4_R3_mode1}
	\caption{Total deformation results for mode $\psi = 1$.}
	\label{fig:4_R3_mode1}
\end{figure}

From above, it can be seen that a load factor of $\lambda = 1668$ was calculated for $t= 0.418\Unit{in}$. This value is within about $21.2\%$ of the expected analytical results.

\subsection{Parametric Study}

For $t\in [0.05, 1.05]$, a parametric study was performed. Results for the first, fifth and tenth mode (i.e. $\psi = 1, 5, 10$) are plotted against the thickness range of interest in Figure~\ref{fig:4_R3_modesweep} below, as per \cite{PYTHON} script in Appendix~\ref{appendix:a4}. The required critical pressure of $p =1376\Unit{psi}$ is also shown for reference.

\begin{figure}[H]
	\centering
	\includegraphics[scale=0.75]{4_R3_modesweep}
	\caption{Parametric sweep results for various $t, \psi$.}
	\label{fig:4_R3_modesweep}
\end{figure}

\begin{figure}[H]
	\centering
	\includegraphics[scale=0.75]{4_R3_comp}
	\caption{Comparison of analytical and FEA results.}
	\label{fig:4_R3_comp}
\end{figure}


The main conclusion which can be drawn for the above results confirms those from Section~\ref{section:3_buckle} that buckling will not be the primary failure mode due to the smaller thicknesses required to attain a critical pressure of $p'$ as calculated in Equation~\ref{eq:2_preq}.

\section{Run 4: Flanges}
\label{section:4_R4}
\subsection{Model}

In hopes of finding a reasonable estimate for sizing the drum assembly's flanges, a final FEA run is completed. Figure~\ref{fig:4_R4_mesh} below shows the mesh for the modeled assembly. Note that the flanges are also modeled as thin surfaces. Drum barrel has thickness of 0.625 in which will be held constant in this section.
\begin{figure}[H]
	\centering
	\includegraphics[scale=0.5]{4_R4_mesh}
	\caption{Drum assembly mesh.}
	\label{fig:4_R4_mesh}
\end{figure}

\subsection{Boundary Conditions}

Figure~\ref{fig:4_R4_BC} shows uniform external pressure of $1376 \Unit{psi}$ and fixed circular region on flange face of 5 in.
\begin{figure}[H]
	\centering
	\includegraphics[scale=0.5]{4_R4_BC}
	\caption{Uniform pressure and fixed flange region.}
	\label{fig:4_R4_BC}
\end{figure}


\subsection{Results}

Figure~\ref{fig:4_R4_res2} Figure~\ref{fig:4_R4_res1}. Note results shown for flanges of thickness 0.125 in.

\begin{figure}[H]
	\centering
	\includegraphics[scale=0.5]{4_R4_res2}
	\caption{Drum assembly stress results.}
	\label{fig:4_R4_res2}
\end{figure}
\begin{figure}[H]
	\centering
	\includegraphics[scale=0.5]{4_R4_res1}
	\caption{Flange stress results for $t=0.125 \Unit{in}$.}
	\label{fig:4_R4_res1}
\end{figure}

Comments, lots of deformation.\\

\subsection{Parametric Study}

A parametric study was ran with $t \in [0.0625, 1.00]$. This sweep results are shown below in Figure~\ref{fig:4_R4_sweep} \cite{EXCEL}. The allowable stress limit of Equation~\ref{eq:2_sigall} of $23,333 \Unit{psi}$ is also shown for reference.

\begin{figure}[H]
	\centering
	\includegraphics[]{4_R4_sweep}
	\caption{Results of parametric study for FEA Run 4.}
	\label{fig:4_R4_sweep}
\end{figure}

Upon completing this highly simplified simulation, it would appear that a thickness of $t\approx 0.25 \Unit{in}$ is the optimal candidate for the thickness of the flanges, given that a thin circular end of 5 in is held fixed. The reason for the increase in flange stress with thickness could be a result of the effects of the fixed end.


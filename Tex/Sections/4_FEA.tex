In this section, the results from previous sections will be compared with those in an ANSYS Mechanical finite element analysis (FEA) software \cite{ANSYS}.

\section{Model}

Using the parameters from Table~\ref{table:prelim_params} , the drum geometry was modeled as a thin surface to save computational cost and time. The geometry and coordinate system adopted in this foregoing analysis are shown below in Figure~\ref{fig:4_geom}.

\begin{figure}[H]
	\centering
	\includegraphics[scale=0.5]{4_geom}
	\caption{ANSYS geometry and coordinate system overview.}
	\label{fig:4_geom}
\end{figure}

In this model, the drum's thin surface has a parameterized inward thickness. This variable will be explored to determine how the maximum stress in the drum varies. The initial mesh is also shown below in \ref{fig:4_mesh}

\begin{figure}[H]
	\centering
	\includegraphics[scale=0.5]{4_mesh}
	\caption{Initial coarse mesh.}
	\label{fig:4_mesh}
\end{figure}

This mesh will be refined in order to achieve convergence on maximum results.


\section{Run 2}

The second run continues on the results determined from Run 1.

\subsection{Boundary Conditions}

Both of the drum's ends are simply supported (i.e. BCs as per \ref{eq:2_endBC} ). In other words, simple supports carry no end moments.\\

Figure~\ref{fig:4_R2_tens} shows the location of applied tether tension of $11,525\ lbs.$ or $51,264\ N$.
\begin{figure}[H]
	\centering
	\includegraphics[scale=0.5]{4_R2_tens}
	\caption{Remote force location in ANSYS model.}
	\label{fig:4_R2_tens}
\end{figure}

Figure~\ref{fig:4_R2_pvar} shows the variable Capstan pressure applied as per results from \ref{subsection:alt} (see Figure~\ref{fig:4_R2_pvarplot} for $p(z)$ ). Note again that the boundary conditions are for $\mu=0.05$ , which is essentially the worst case scenario.

\begin{figure}[H]
	\centering
	\includegraphics[scale=0.5]{4_R2_pvar}
	\caption{Variable Capstan pressure.}
	\label{fig:4_R2_pvar}
\end{figure}
\begin{figure}[H]
	\centering
	\includegraphics[scale=0.6]{4_R2_pvarplot}
	\caption{Variable Capstan pressure as a function of longitudinal coordinate $z$.}
	\label{fig:4_R2_pvarplot}
\end{figure}


\subsection{Results}

After a mesh refinement, total deformation results are shown below in Figure~\ref{fig:4_R2_def_mu05}. The maximum deformation seen on the top surface of the drum in the $y$ direction with respect to $z$ are also shown in Figure~\ref{fig:4_R2_topdefplotmu05}.

\begin{figure}[H]
	\centering
	\includegraphics[scale=0.5]{4_R2_def_mu05}
	\caption{Total deformation results.}
	\label{fig:4_R2_def_mu05}
\end{figure}
\begin{figure}[H]
	\centering
	\includegraphics[scale=0.6]{4_R2_topdefplotmu05}
	\caption{Total deformation in $y$ as a function of $z$.}
	\label{fig:4_R2_topdefplotmu05}
\end{figure}

Comments, lots of deformation.\\

Again, post mesh refinement yields the maximum stress results as per Figure~\ref{fig:4_R2_stress_mu05}.

\begin{figure}[H]
	\centering
	\includegraphics[scale=0.5]{4_R2_stress_mu05}
	\caption{Total stress results.}
	\label{fig:4_R2_stress_mu05}
\end{figure}

Comments, bad stress.


\subsection{Parametric Study}

A parameteric study was ran with $0.15 \geq t \geq 1.05$ and $0.05 \geq \mu \geq 0.50$. This sweep is shown below in Figure~\ref{fig:4_R2_sweep}. The allowable stress limit of $\sigma_{allow}=\sigma_{y}/SF = 23,333\ psi$ is also shown for reference.

\begin{figure}[H]
	\centering
	\includegraphics[scale=0.75]{4_R2_sweep}
	\caption{Results of parametric study for FEA Run 2.}
	\label{fig:4_R2_sweep}
\end{figure}

From the above plot, it appears as though a thickness of $t geq 0.50\ in$ would appear to be sufficient.



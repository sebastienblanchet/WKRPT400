This section uses ANSYS Mechanical FEA software \cite{ANSYS} to validate analytical findings from previous sections. Please note that all results presented in the foregoing section utilizes USC units however, important results will be converted to their respective SI units. The overall ANSYS geometry is first presented followed by a series of FEA runs which are summarized below (see Table~\ref{table:4_runs}).

\begin{table}[H]
  \centering
  \caption{Labeling of FEA runs.}
    \begin{tabular}{cll}
    \textbf{Run } & \textbf{Focus} & \textbf{Description}\\
    \hline
    1A    & Drum Stress & Fixed ends, uniform $p$\\
    1B    & Drum Stress & Simply supported ends, uniform $p$ \\
    2A    & Drum Stress & Fixed ends, Capstan $p, z=24.5$, various $\mu$ \\
    2B    & Drum Stress & Simply supported ends, Capstan $p, z=24.5$, various $\mu$ \\
    2C    & Drum Stress & Fixed ends, Capstan $p, z=0$, various $\mu$ \\
    2D    & Drum Stress & Simply supported ends, Capstan $p, z=0$, various $\mu$ \\
    3     & Drum Buckling & Simply supported ends, uniform $p$ \\
    4     & Flange Stress & Fixed ends, uniform $p$ \\
    \end{tabular}%
  \label{table:4_runs}%
\end{table}%


\section{Model}

Both $D$ and $L$ parameters from Table~\ref{table:prelim_params} are used to model the drum barrel's geometry. The coordinate system adopted in the foregoing analysis is shown below in Figure~\ref{fig:4_geom}. 

\begin{figure}[H]
	\centering
	\includegraphics[scale=0.5]{4_geom}
	\caption{Geometry and coordinate system overview.}
	\label{fig:4_geom}
\end{figure}

This model utilizes a thin cylindrical surface with inwards parameterized thickness to save computational time. A series of parametric studies will be performed determine how maximum stresses vary with $t$. Note that flanges are not shown in this model as they will be added in the final run.\\

The thin surface also allows for a simple initial ANSYS generated coarse mesh (see Figure~\ref{fig:4_mesh}).

\begin{figure}[H]
	\centering
	\includegraphics[scale=0.5]{4_mesh}
	\caption{Initial model coarse mesh.}
	\label{fig:4_mesh}
\end{figure}

Numerical accuracy of results will be assured by performing mesh refinement when possible in all foregoing simulations.

\section{Run 1: Uniform Pressure}
\label{section:4_R1}

The first FEA run will explore the simply uniformly loaded drum with both fixed and free ends. 

\subsection{Boundary Conditions}
\label{subsection:R1BC}

An initial simulation will be performed with fixed ends (BC as per Equation~\ref{eq:2_fixedBC}) and $t=0.500\Unit{in}$. Figure~\ref{fig:4_R1_p} shows the uniform external pressure of $1376 \Unit{psi}$ or $9.484 \Unit{MPa}$ as per \ref{eq:2_preq}.
\begin{figure}[H]
	\centering
	\includegraphics[scale=0.5]{4_R1_p}
	\caption{Uniform external pressure on model.}
	\label{fig:4_R1_p}
\end{figure}

\subsection{Results}

A mesh refinement is completed and yields the equivalent maximum stress results in Figure~\ref{fig:4_R1_stress} below. Note results are shown for $t=0.500\Unit{in}$.

\begin{figure}[H]
	\centering
	\includegraphics[scale=0.5]{4_R1_stress}
	\caption{Maximum equivalent stress results for Run 1.}
	\label{fig:4_R1_stress}
\end{figure}

It is apparent from the above figure that the fixed ends are carrying most of load. In hopes of better understanding how this uniform pressure results in stress, a parametric study will is performed by varying the drum thickness $t$ for a cylinder with both fixed and simply supported ends.


\subsection{Parametric Study}

A parametric study is perforned with a range of $t \in [0.50, 1.75]$ for both fixed and simply supported ends. The results from this simulation are shown below in Figure~\ref{fig:4_R1_sweep} \cite{EXCEL}. The allowable stress limit of $23,333 \Unit{psi}$ (Equation~\ref{eq:2_sigall}) is also shown for reference.

\begin{figure}[H]
	\centering
	\includegraphics[scale=0.75]{4_R1_sweep}
	\caption{Results the parametric sweep for Run 1.}
	\label{fig:4_R1_sweep}
\end{figure}

From the above plot, a thicknesses of $1.25 \Unit{in}$ and $0.875\Unit{in}$ are required for uniformly loaded cylinder with fixed and simply supported ends respectively.

\section{Run 2: Capstan Pressure}
\label{section:4_R2}
The second run continues on the results determined from Run 1. The results presented in this foregoing section will focus on a barrel thickness of $0.50\ in$.

\subsection{Boundary Conditions}

Both of the drum's ends are simply supported (i.e. BCs as per \ref{eq:2_endBC} ). In other words, simple supports carry no end moments.\\

Figure~\ref{fig:4_R2_tens} shows the location of applied tether tension of $11,525\ lbs.$ or $51,264\ N$.
\begin{figure}[H]
	\centering
	\includegraphics[scale=0.5]{4_R2_tens}
	\caption{Remote force location in ANSYS model.}
	\label{fig:4_R2_tens}
\end{figure}

Figure~\ref{fig:4_R2_pvar} shows the variable Capstan pressure applied as per results from \ref{subsection:alt} (see Figure~\ref{fig:4_R2_pvarplot} for $p(z)$ ). Note again that the boundary conditions are for $\mu=0.05$ , which is essentially the worst case scenario. Note results shown for $t=0.50\ in$

\begin{figure}[H]
	\centering
	\includegraphics[scale=0.5]{4_R2_pvar}
	\caption{Variable Capstan pressure.}
	\label{fig:4_R2_pvar}
\end{figure}
\begin{figure}[H]
	\centering
	\includegraphics[scale=0.6]{4_R2_pvarplot}
	\caption{Variable Capstan pressure as a function of longitudinal coordinate $z$.}
	\label{fig:4_R2_pvarplot}
\end{figure}


\subsection{Results}

After a mesh refinement, total deformation results are shown below in Figure~\ref{fig:4_R2_def_mu05}. The maximum deformation seen on the top surface of the drum in the $y$ direction with respect to $z$ are also shown in Figure~\ref{fig:4_R2_topdefplotmu05}.

\begin{figure}[H]
	\centering
	\includegraphics[scale=0.5]{4_R2_def_mu05}
	\caption{Total deformation results.}
	\label{fig:4_R2_def_mu05}
\end{figure}
\begin{figure}[H]
	\centering
	\includegraphics[scale=0.6]{4_R2_topdefplotmu05}
	\caption{Total deformation in $y$ as a function of $z$.}
	\label{fig:4_R2_topdefplotmu05}
\end{figure}

Comments, lots of deformation.\\

Again, post mesh refinement yields the maximum stress results as per Figure~\ref{fig:4_R2_stress_mu05}.

\begin{figure}[H]
	\centering
	\includegraphics[scale=0.5]{4_R2_stress_mu05}
	\caption{Total stress results.}
	\label{fig:4_R2_stress_mu05}
\end{figure}

Comments, bad stress.


\subsection{Parametric Study}

A parametric study was ran with  $t\in [0.15, 1.05]$ and $\mu =0.05, 0.50$. This sweep results are shown below in Figure~\ref{fig:4_R2_sweep} \cite{EXCEL}.  The allowable stress limit of Equation~\ref{eq:2_sigall} of $23,333 \Unit{psi}$ is also shown for reference.

\begin{figure}[H]
	\centering
	\includegraphics[]{4_R2_sweep_FE_SSE}
	\caption{Results of parametric study for FEA Run 2.}
	\label{fig:4_R2_sweep}
\end{figure}

The main take away from this above sweep is that for a Capstan pressure profile beginning at the center of the drum, the end conditions do not matter. This can be validated through thin shell theory (i.e $\beta x geq 6$).\\

From the above plot, it appears as though a thickness of $t \geq 0.55 \Unit{in}$ would appear to be sufficient.\\

To investigate the effects of the end conditions, the Capstan pressure was applied from $0 \leq z \leq 24.5$. See figures 



\begin{figure}[H]
	\centering
	\includegraphics[scale=0.5]{4_R2_BC_c}
	\caption{Variable Capstan pressure.}
	\label{fig:4_R2_BC_c}
\end{figure}

\begin{figure}[H]
	\centering
	\includegraphics[scale=0.5]{4_R2_stress_c}
	\caption{Total stress results.}
	\label{fig:4_R2_stress_c}
\end{figure}


\begin{figure}[H]
	\centering
	\includegraphics[]{4_R2_sweep_c}
	\caption{Results of parametric study for FEA Run 2.}
	\label{fig:4_R2_sweep}
\end{figure}

\section{Run 3: Eigenvalue Buckling}
\label{section:4_R3}
Very unconservative \\

Buckling mode shapes do not represent actual displacements but help you to visualize how a part or an assembly deforms when buckling. \cite{ANSYS}. \\

Load factors $\lambda$ highly dependent on pre-stressed state, either linear on non linear.\\

In this case linear buckling is utilized for its simplicity. Must have coupled static structural analysis (see Figure~\ref{fig:4_R3_wb}.

\begin{figure}[H]
	\centering
	\includegraphics[scale=0.5]{4_R3_wb}
	\caption{Workbench project schematic.}
	\label{fig:4_R3_wb}
\end{figure}

ANSYS takes in the pre-stressed state and returns load factors $\lambda$ which are defined as follows in Equation~\ref{eq:4_loadfactor}.
\begin{equation}
	\label{eq:4_loadfactor}
	p' = \lambda \ p_0
\end{equation}

For this reason, a unit load pressure of $p_0 = 1\ psi$ will be applied. The boundary conditions will follow those of Section~\ref{subsection:R1BC}.

\subsection{Boundary Conditions}

Boundary conditions of coupled static structural analysis. Ends simply supported (Equation~\ref{eq:2_endBC}) and a unit pressure load applied to external surface (see Figure~\ref{fig:4_R3_BC}). Furthermore, for comparison of critical buckling calculations in Section~\ref{section:3_buckle}, a thickness of $t=0.418\Unit{in}$ was modeled. 

\begin{figure}[H]
	\centering
	\includegraphics[scale=0.5]{4_R3_BC}
	\caption{Boundary conditions for Eigenvalue Buckling analysis.}
	\label{fig:4_R3_BC}
\end{figure}

\subsection{Results}

Unfortunately, ANSYS is unable to perform a mesh refinement study for a coupled static structural and eigenvalue buckling analysis hence, a much finer initial mesh than that shown in Figure~\ref{fig:4_mesh} was used to assure reasonably accurate results on the first iteration.\\

Results are shown below for the first eigenvalue or mode $\psi = 1$ below in Figure~\ref{fig:4_R3_mode1}.
\begin{figure}[H]
	\centering
	\includegraphics[scale=0.5]{4_R3_mode1}
	\caption{Total deformation results for mode $\psi = 1$.}
	\label{fig:4_R3_mode1}
\end{figure}

From above, it can be seen that a load factor of $\lambda = 1668$ was calculated for $t= 0.418\Unit{in}$. This value is within about $21.2\%$ of the expected analytical results.

\subsection{Parametric Study}

For $t\in [0.05, 1.05]$, a parametric study was performed. Results for the first, fifth and tenth mode (i.e. $\psi = 1, 5, 10$) are plotted against the thickness range of interest in Figure~\ref{fig:4_R3_modesweep} below, as per \cite{PYTHON} script in Appendix~\ref{appendix:a4}. The required critical pressure of $p =1376\Unit{psi}$ is also shown for reference.

\begin{figure}[H]
	\centering
	\includegraphics[scale=0.75]{4_R3_modesweep}
	\caption{Parametric sweep results for various $t, \psi$.}
	\label{fig:4_R3_modesweep}
\end{figure}

\begin{figure}[H]
	\centering
	\includegraphics[scale=0.75]{4_R3_comp}
	\caption{Comparison of analytical and FEA results.}
	\label{fig:4_R3_comp}
\end{figure}


The main conclusion which can be drawn for the above results confirms those from Section~\ref{section:3_buckle} that buckling will not be the primary failure mode due to the smaller thicknesses required to attain a critical pressure of $p'$ as calculated in Equation~\ref{eq:2_preq}.

\section{Run 4: Flanges}
\label{section:4_R4}
\subsection{Model}

In hopes of finding a reasonable estimate for sizing the drum assembly's flanges, a final FEA run is completed. Figure~\ref{fig:4_R4_mesh} below shows the mesh for the modeled assembly. Note that the flanges are also modeled as thin surfaces. Drum barrel has thickness of 0.625 in which will be held constant in this section.
\begin{figure}[H]
	\centering
	\includegraphics[scale=0.5]{4_R4_mesh}
	\caption{Drum assembly mesh.}
	\label{fig:4_R4_mesh}
\end{figure}

\subsection{Boundary Conditions}

Figure~\ref{fig:4_R4_BC} shows uniform external pressure of $1376 \Unit{psi}$ and fixed circular region on flange face of 5 in.
\begin{figure}[H]
	\centering
	\includegraphics[scale=0.5]{4_R4_BC}
	\caption{Uniform pressure and fixed flange region.}
	\label{fig:4_R4_BC}
\end{figure}


\subsection{Results}

Figure~\ref{fig:4_R4_res2} Figure~\ref{fig:4_R4_res1}. Note results shown for flanges of thickness 0.125 in.

\begin{figure}[H]
	\centering
	\includegraphics[scale=0.5]{4_R4_res2}
	\caption{Drum assembly stress results.}
	\label{fig:4_R4_res2}
\end{figure}
\begin{figure}[H]
	\centering
	\includegraphics[scale=0.5]{4_R4_res1}
	\caption{Flange stress results for $t=0.125 \Unit{in}$.}
	\label{fig:4_R4_res1}
\end{figure}

Comments, lots of deformation.\\

\subsection{Parametric Study}

A parametric study was ran with $t \in [0.0625, 1.00]$. This sweep results are shown below in Figure~\ref{fig:4_R4_sweep} \cite{EXCEL}. The allowable stress limit of Equation~\ref{eq:2_sigall} of $23,333 \Unit{psi}$ is also shown for reference.

\begin{figure}[H]
	\centering
	\includegraphics[]{4_R4_sweep}
	\caption{Results of parametric study for FEA Run 4.}
	\label{fig:4_R4_sweep}
\end{figure}

Upon completing this highly simplified simulation, it would appear that a thickness of $t\approx 0.25 \Unit{in}$ is the optimal candidate for the thickness of the flanges, given that a thin circular end of 5 in is held fixed. The reason for the increase in flange stress with thickness could be a result of the effects of the fixed end.


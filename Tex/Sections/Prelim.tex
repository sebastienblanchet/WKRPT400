\label{chapt:prelim}

In this section, the methods used for the numerical approach are discussed in detail.

%%----------------------------------------------------------------------------------------------------------------------
\section{Basics: Deflection}

The first equation that is presented is the main relation between the internal bending moment on a beam and its curvature by Equation~\ref{eq:Mcurve} as per \citep{nisbett2014shigley}. 

% First equation of basic statics
\begin{equation}
	\label{eq:Mcurve}
%	% must use aligned to get multi line equation	
%	\begin{aligned}
%		\Sigma F_i=0 \\
%		\Sigma M_i=0 
%	\end{aligned}
	\frac{1}{\rho}=\frac{M}{EI}
\end{equation}

Where $\frac{1}{\rho}$ is the radius of curvature defined as \ref{eq:curve}

\begin{equation}
	\label{eq:curve}
	\frac{1}{\rho}=\frac{d^2y/dx^2}{\left( 1 +(dy/dx)^2 \right)^frac{3}{2}} \approx \frac{d^2y}{dx^2}
\end{equation}

\subsection{Relations}

With \ref{eq:curve}, relations based on deflection $y(x)$ for slope \ref{eq:slope}, moment \ref{eq:mmt}, shearing force \ref{eq:shr} and load intensity \ref{eq:loadintens} as per \citep{nisbett2014shigley}.

\begin{equation}
	\label{eq:slope}
	\theta(x) = \frac{dy}{dx}
\end{equation}

\begin{equation}
	\label{eq:mmt}
	M(x) = EI\ \frac{d^2y}{dx^2}
\end{equation}

\begin{equation}
	\label{eq:shr}
	V(x) = EI\ \frac{d^3y}{dx^3}
\end{equation}

\begin{equation}
	\label{eq:loadintens}
	q(x) = EI\ \frac{d^4y}{dx^4}
\end{equation}

\subsection{Boundary Conditions}

From these above equations, we may present boundary conditions as follows assuming that the boundary in question is located at $x_0$.\\

\textbf{Free Ends:}\\
\begin{equation}
	\label{eq:freeBC}
	\begin{aligned}
	y(x_0) = y_0 \\
	\theta(x_0)= \theta_0\\
	M(x_0) = 0\\
	V(x_0) = 0 
	\end{aligned}
\end{equation}

\textbf{End Supported:}\\
\begin{equation}
	\label{eq:freeBC}
	\begin{aligned}
	y(x_0)=y_0 \\
	\theta(x_0)=\theta_0\\
	M(x_0)=M_0\\
	V(x_0) =V_0 
	\end{aligned}
\end{equation}

\textbf{Fixed:}\\
\begin{equation}
	\label{eq:freeBC}
	\begin{aligned}
	y(x_0)=0 \\
	\theta(x_0)=0\\
	M(x_0)=M_0\\
	V(x_0) =V_0 
	\end{aligned}
\end{equation}

In the following section, these preliminary relations will be used to further expand on more complex analysis approaches.

%Example of using table
%%\subsection{Table}
%%
%%This was used to get the results in the following Table~\ref{table:testtbl}.
%%	
%%\begin{table}[ht]
%%	\caption{Test table 1}
%%	\centering
%%	\begin{tabular}{c c c}
%%		%heading
%%		\hline \hline  Case & Method \#1 & Method \#2 \\[0.5ex]
%%		\hline
%%		%Main table body content	
%%		1                   & 10         & 50         \\
%%		2                   & 20         & 100        \\ [1ex]
%%		\hline
%%	\end{tabular}
%%	\label{table:testtbl}
%%\end{table}

%----------------------------------------------------------------------------------------------------------------------
\section{Capstan Equation}
What the results in the previous section reveal is that the solution lies in the added level of complexity in this problem. It is required to understand how one wrap of tether in tension results in an external pressure q applied to the outer surface of the cylinder.\\

After much research into this problem, the solution reveals itself into the derivation of the well-known Capstan equation \ref{eq:Capstan}.

\begin{equation}
	\label{eq:Capstan}
	T_2 = T_1 e^{-\mu\theta}
\end{equation}


From this equation, the free body diagram that leads to this solution is presented below in Figure~\ref{fig:Capstan}. 

\begin{figure}[!htbp]
	\centering
	\includegraphics[width=0.6\textwidth]{Capstan}
	\caption{Free body diagram of differential capstan problem.}
	\label{fig:Capstan}
\end{figure}

Based on this, what is of question is how to extrapolate a pressure $q$ from the applied tension $T$. This can be done performing a force balance in the radial direction which reduces to Equation~\ref{eq:CapstanSigRad}.

\begin{equation}
	\label{eq:CapstanSigRad}
	dN-(T+dT)\sin \frac{d\theta}{2}+T\sin \frac{d\theta}{2}= 0
\end{equation}

By assuming a small chance in angle, the substitution of $\sin \theta \approx \theta$ can be made. Applying this relation to \ref{eq:CapstanSigRad} leaves \ref{eq:diffNormal}.

\begin{equation}
	\label{eq:diffNormal}
	dN = T d\theta
\end{equation}

From the above equation, it is now clear that the overall normal force caused by the tension in the cable $T$ can be solved for by integration or as follows \ref{eq:CapstanNorm}

\begin{equation}
	\int_0^N dN =\int_0^{2\pi} T d\theta
\end{equation}

\begin{equation}
	\label{eq:CapstanNorm}
	N=2\pi T	
\end{equation}

With this resultant normal force over one revolution of tether, it is now apparent that the pressure $q$ is simply \ref{eq:CapstanPres}

\begin{equation}
	\label{eq:CapstanPres}
	q=\frac{N}{A}=\frac{2\pi T}{d\pi D}=\frac{2T}{Dd}
\end{equation}

The result concluded in this section will reveal to be very important in the comming sections.

%----------------------------------------------------------------------------------------------------------------------
\section{Theory of Cylindrical Shells}

As per Timoshenko's book \citep{timoshenko1959theory}, this section will cover the method of approximating the cylinder as a long thin shell. The following coordinate system is presented as per Figure~\ref{fig:CoordSyst} below.

\begin{figure}[!htbp]
	\centering
	\includegraphics[width=0.6\textwidth]{CoordSyst}
	\caption{Coordinate system adopted for derivation of DEs.}
	\label{fig:CoordSyst}
\end{figure}

Based on this, the following differential equations are presented as a pressure balance knowing that the differential area can be represented as Equation~\ref{eq:diffsurf}
 
\begin{equation}
	\label{eq:diffsurf}
	dA = dS\ dx = a\ d\varphi \ dx   
\end{equation}

The equations of equilibrium may be written as a force projection about the x and z axis and momment balance about y as Equations ~\ref{eq:eqbrm_x}, ~\ref{eq:eqbrm_z}, ~\ref{eq:eqbrm_y} respectively.

\begin{equation}
	\label{eq:eqbrm_x}
	\frac{dN_x}{dx}\ a\ d\varphi \ dx = 0
\end{equation}

\begin{equation}
	\label{eq:eqbrm_z}
	\frac{dQ_x}{dx}\ a\ d\varphi \ dx+ N_\varphi \ a\ d\varphi \ dx +Z\ a\ d\varphi \ dx= 0
\end{equation}

\begin{equation}
	\label{eq:eqbrm_y}
	\frac{dM_x}{dx}\ a\ d\varphi \ dx- Q_x\ a\ d\varphi \ dx= 0
\end{equation}

First looking at \ref{eq:eqbrm_x}, simplifying this equation and taking the integral with respect to $x$ will leave \ref{eq:eqbrm_x2}. 

\begin{equation}
	\label{eq:eqbrm_x2}
	N_x = C = 0 
\end{equation}

This above equation simply states that the effects of bending due to the axial forces will be neglected.\\

Similarly with \ref{eq:eqbrm_z} and \ref{eq:eqbrm_y}, simplifications will lead to \ref{eq:eqbrm_z2} and \ref{eq:eqbrm_y2} respectively.
\begin{equation}
	\label{eq:eqbrm_z2}
	\frac{dQ_x}{dx}+\frac{1}{a}\ N_\varphi = -Z
\end{equation}

\begin{equation}
	\label{eq:eqbrm_y2}
	\frac{dM_x}{dx}- Q_x= 0
\end{equation} 

With these equations and using strain relations from Hooke's law, the fine DE for displacement will be determined. In Equation ~\ref{eq:strain_xphi}, the relations between displacement and strain are presented.

\begin{equation}
	\label{eq:strain_xphi}
	\begin{aligned}
		\epsilon_x = \frac{du}{dx}      \\
		\epsilon_\varphi = -\frac{w}{a} 
	\end{aligned}
\end{equation}

From Hooke's law $N_x$ may be also written as Equation~\ref{eq:Hookes_Nx}. Substituting ~\ref{eq:strain_xphi} will leave the final simplification.
\begin{equation}
	\label{eq:Hookes_Nx}
	N_x = \frac{Eh}{1-\nu^2}\ \left( \epsilon_x + \nu \epsilon_\varphi \right) =  \frac{Eh}{1-\nu^2}\ \left( \frac{du}{dx} -\nu \ \frac{w}{a} \right)
\end{equation} 

%%tilde is UNBREAKABLE SPACE
Solving Equation~\ref{eq:Hookes_Nx} using ~\ref{eq:eqbrm_x2} leaves \ref{eq:Nx_simpl}.
\begin{equation}
	\label{eq:Nx_simpl}
	\frac{du}{dx} =  \nu \ \frac{w}{a}
\end{equation} 

Similarly, with $N_\varphi$, again applying \ref{eq:strain_xphi} leaves \ref{eq:Nphi_simpl}.
\begin{equation}
	\label{eq:Hookes_Nphi}
	N_\varphi = \frac{Eh}{1-\nu^2}\ \left( \epsilon_\varphi + \nu \epsilon_x \right) = \frac{Eh}{1-\nu^2}\  \left( -\frac{w}{a}+\nu \ \frac{du}{dx} \right)
\end{equation} 

\begin{equation}
	\label{eq:Nphi_simpl}
	N_\varphi = - \frac{Ehw}{a}
\end{equation}

As a result of no change in curvature in the $\varphi$ direction, we know that $\frac{dM_\varphi}{d\varphi}= 0$ hence no change in the circumferential moments. This relation is translated to axial momments $M_x$ with \ref{eq:Mphix}.

\begin{equation}
	\label{eq:Mphix}
	\begin{aligned}
	M_\varphi = \nu M_x\\
	M_x = -D \frac{d^2w}{dx^2}
	\end{aligned}
\end{equation}

Where $D$ is the flexural rigidity of the shell \ref{eq:flexrig}.

%\begin{equation}
%	\label{eq:EQNNAME}
%	EQUN
%\end{equation} 

\begin{equation}
	\label{eq:flexrig}
	D = \frac{Eh^3}{12(1-\nu^3)}
\end{equation}

Simplifying \ref{eq:eqbrm_z2} to get $Q_x = \frac{dM_x}{dx}$, the following is put in \ref{eq:eqbrm_y2} to get \ref{eq:de1}

\begin{equation}
	\label{eq:de1}
	\begin{aligned}
	\frac{d^2M_x}{dx^2}+\frac{1}{a} \ N_\varphi = -Z\\
	D\ \frac{d^4w}{dx^4}+\frac{Eh}{a^2} \ w = Z\\
	\frac{d^4w}{dx^4}+\beta^4 \ w = \frac{Z}{D}
	\end{aligned}
\end{equation} 

Where $\beta^4$ is some parameter defined as \ref{eq:betaquad}.

\begin{equation}
	\label{eq:betaquad}
	\beta^4 = \frac{Eh}{4a^2D}= \frac{3(1-\nu^2)}{a^2h^2}
\end{equation}

The solution to this common fourth order, linear, non-homogeneous, ordinary differential equation in \ref{eq:de1} has is known to be Equation \ref{eq:solnDE}

\begin{equation}
	\label{eq:solnDE}
	w(x)=e^{\beta x} \left(C_1 \cos \beta x +C_2 \sin \beta x \right)+e^{-\beta x} \left(C_3 \cos \beta x +C_4 \sin \beta x \right) +f(x)
\end{equation}

Where $ C_1, C_2, C_3, C_4$ are intgration constants to be solved based on BCs and $f(x)$ is the particular solution to the ODE.

In the following section, the solution to this differential equation will be explored.

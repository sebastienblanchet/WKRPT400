
In this section, well known standards will be investigated.

\begin{itemize}
    \item American Society of Mechanical Engineers:
	    \begin{itemize}[label=$\bullet$]
	       	\item Section III: Part D, 2013	\citep{ASMEbvpcIID}
	    	\item Section VII: Division 1, 2015 \citep{ASMEbvpcVII1}
	    	\item Section VII: Division 2, 2015 \citep{ASMEbvpcVII2}
	    \end{itemize}
    \item European Standard EN
        \begin{itemize}[label=$\bullet$]
	       	\item EN 13445-3:2014
	    \end{itemize}
\end{itemize}

\section{ASME BPVC}

\subsection{Section VII}
As per UG-28 of \citep{ASMEbvpcVII1}, the following procedure was used to calculate the required thickness as per ASME-BVPC-VII-1 code.\\

The following list of steps were following during this process

\begin{enumerate}
	\item Assume initial thickness value of $t$
	\item Calculate $D_o/t$ ratio and assure $D_o/t \geq 10$.
	\item Calculate $L/D_o$ ratio, if $L/D_o \geq 50 \Rightarrow =50$ or  $L/D_o \leq 0.05 \Rightarrow =0.05$
	\item With above ratios, go to Figure G of \citep{ASMEbvpcIID} and get value for $A$
	\item With $A$ from above go to chart CS-2 $\because S_y \geq 30 \ ksi$ to get $B$
	\item Using $B$ use Equation \ref{eq:VII_1_stp6} to calculate allowable pressure $p_a$:
		\begin{equation}
			\label{eq:VII_1_stp6}
			p_a = \frac{4B}{3 \left(\frac{D_o}{t} \right)}
		\end{equation}
	\item Check if $p_a \geq p_{req}$ of \ref{eq:
	
\end{enumerate}

\section{European European Standard EN}

\section{Comparison}


\section{Result Summary}

Based on all findings from this report, a summary is shown below in Table~\ref{table:5_sum}.

\begin{table}[H]
  \centering
  \caption{Tabulated summary of all report results.}
    \begin{tabular}{clccll}
          &       & \multicolumn{2}{c}{\textbf{Results}} &       &  \\
    \textbf{Section} & \textbf{Method} & \textbf{mm} & \textbf{in} & \textbf{BC} & \textbf{Pressure } \\
    \midrule
          \ref{section:2_VIII1}& ASME BPVC VIII-1 & $43.2$ & $1.680$ & Fixed & Uniform \\
          \ref{section:2_VIII2}& ASME BPVC VIII-2 & $23.4$ & $0.920$ & Fixed & Uniform \\
          \ref{section:2_EN}& EN 13445-3 & $32.8$ & $1.290$ & Fixed & Uniform \\
          \ref{section:2_DNV}& DNV-OS-D101 & $36.6$ & $1.440$ & Fixed & Uniform \\
          \ref{section:2_TWPV}& TWPV Hoop Stress & $21.0$ & $0.825$ & Simply supported & Uniform \\
          \ref{section:3_roark}& Roarks & $42.8$ & $1.686$ & Fixed & Uniform \\
          \ref{section:3_buckle}& Roarks Buckling & $10.6$ & $0.418$ & Simply supported & Uniform \\
          \ref{section:4_R1}& FEA Run 1A & $30.5$ & $1.200$ & Fixed & Uniform \\
          \ref{section:4_R1}& FEA Run 1B & $22.2$ & $0.875$ & Simply supported & Uniform \\
          \ref{section:4_R2}& FEA Run 2A & $14.0$ & $0.550$ & Fixed & Capstan $z=24.5$ \\
          \ref{section:4_R2}& FEA Run 2B & $14.0$ & $0.550$ & Simply supported & Capstan $z=24.5$\\
          \ref{section:4_R2}& FEA Run 2C & $16.5$ & $0.650$ & Fixed & Capstan $z=0$ \\
          \ref{section:4_R2}& FEA Run 2D & $14.0$ & $0.550$ & Simply supported & Capstan $z=0$  \\
          \ref{section:4_R3}& FEA Run 3 & $9.5$ & $0.375$ & Simply supported & Uniform \\
    \end{tabular}
  \label{table:5_sum}
\end{table}%

It is now required to discuss the feasibility of all above results. The most realistic scenario likely represents that from FEA Run 2, with the simply supported end drum with the Capstan pressure profile.\\

From this, it is realistic to say that a thickness of $t\geq 0.65\ \Unit{in} /\/ 16.5\Unit{mm}$ should be selected. 

\section{Manufacturing}

It must be accounted for that this will have to manufactured with standard available ASME pipe. See Table~\ref{table:5_pipe} for standard pipe available as per \cite{PIPEINFO}. Note that all dimensions are shown in USC as per standard pipe.

\begin{table}[H]
	\caption[Available wall thicknesses for standard pipe.]{Available wall thicknesses for standard pipe.\protect\cite{PIPEINFO}}
	\centering
	\begin{tabular}{ccc}
    \textbf{Thickness} & \textbf{ID} & \textbf{Pipe Schedule} \\
    \midrule
    0.312 & 27.376 & 10 \\
    0.375 & 27.250 & STD \\
    0.500 & 27.000 & 20 \\
    0.625 & 26.750 & 30 \\
    \end{tabular}%
	\label{table:5_pipe}
\end{table}

From above, a Schedule 30, 28 in OD pipe with a wall thickness of 0.625 in must be selected.

\section{Numerical Error}
\label{subsection:5_numerr}
As a result of running a large range of \cite{ANSYS} simulations (i.e. 112 total), the error margins used for numerical convergence were set to be relatively high (i.e. 20\%). This being said, there is likely to be a significant error related to the FEA results which must be understood.


\section{Other Considerations}
Asides from erroneous values attained from software and calculation, there are some further factors to consider. For instance, most analytical methods used in Sections 2 and 3 assume very thin cylindrical shells. A key part of this assumption lies in the fact that bending stresses do not develop axially (see Section~\ref{section:3_shells}) \cite{timoshenko1959theory}. \\

Also, material imperfections resulting in higher stress concentrations. This phenomenon highly affects the results determined from a buckling analysis. It is these imperfections that cause non-linearities prevent most real-world structures from achieving their theoretical elastic buckling strength \cite{ANSYS}.


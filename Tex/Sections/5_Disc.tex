\section{Result Summary}


From this, a $t= X\ in= X\ mm$ should be selected.

\section{Manufacturing}

It must be accounted for that this will have to manufactured with standard available ASME pipe. See Table~\ref{table:5_pipe} for standard pipe available as per \cite{PIPEINFO}. Note that all dimensions are shown in USC as per standard pipe.

\begin{table}[H]
	\caption[Available wall thicknesses for standard pipe.]{Available wall thicknesses for standard pipe.\protect\cite{PIPEINFO}}
	\centering
	\begin{tabular}{ccc}
    \textbf{Thickness} & \textbf{I.D} & \textbf{Pipe Schedule} \\
    \midrule
    0.312 & 27.376 & 10 \\
    0.375 & 27.250 & STD \\
    0.500 & 27.000 & 20 \\
    0.625 & 26.750 & 30 \\
    \end{tabular}%
	\label{table:5_pipe}
\end{table}

\section{Numerical Error}
Numerical error\\


\section{Other Considerations}
Assumptions of thin walled\\

Material imperfections resulting in higher stress concentrations\\

Linear buckling\\
imperfections and nonlinearities prevent most real-world structures from achieving their theoretical elastic buckling strength\\


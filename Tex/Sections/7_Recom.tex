
Based on the report's analysis, several recommendations can be made to refine the results. First, as for boundary conditions, more work should be completed on the flanges. It is how these components behave under various loading that will determine what that will determine the proper drum barrel's end conditions. The initial results serve as a great benchmark however the flange design and analysis should be iterated upon.\\

In terms of buckling, it was mentioned that the linear analysis yields highly unconservative results relative to a nonlinear set-up. Consequently, to further verify and assure that buckling is not the primary mode of failure, a nonlinear analysis should be investigated.\\

%aerstat's aerodynamics
Furthermore, due to the cyclic nature of the changing tether loads, it would also be important to perform a fatigue analysis. This should only be investigated however when the project reaches the commercial phase and the system will require to be functional for years at a time. This ignored for this report as the proposed TMS winch system is set for testing purposes.\\

Finally, to assure that conclusions are indeed safe, a series of proof testing should be completed on the final drum assembly to assure that the system is capable of withstanding loads. This would be the final validation step for analytical and FEA work.
